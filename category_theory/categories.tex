\documentclass[11pt, oneside]{article}
\def\jtitle{Categories}
\def\jlecturer{Richard Borcherds}
\def\jterm{}
\usepackage[course]{jack}

\DeclareMathOperator{\im}{im}

\begin{document}
\maketitle
Based on {Richard Borcherds's YouTube series on categories}\footnote{See \url{https://www.youtube.com/playlist?list=PL8yHsr3EFj51F9XZ\_Ka4bLnQoxTdMx0AL}.} and Saunders Mac Lane's {\underline{Categories for the Working Mathematician}}.\footnote{See \url{https://link.springer.com/book/10.1007\%2F978-1-4757-4721-8}.} \color{black}\cite{CWM}
\tableofcontents
\section{Introduction}
The first paper on category theory, by \href{https://en.wikipedia.org/wiki/Samuel_Eilenberg}{\color{black}Eilenberg} and \href{https://en.wikipedia.org/wiki/Saunders_Mac_Lane}{\color{black}Mac Lane}, was entitled \underline{General Theory}  \underline{of Natural Equivalences} and originally published in 1942.

A \href{https://en.wikipedia.org/wiki/Category_(mathematics)}{\defn{category}}\index{category} is composed of two things: \defn{objects}\index{object} and \defn{morphisms}\index{morphism}. {Categories} are named after their objects. In the category of sets, the morphisms are functions. In the category of groups, the morphisms are homomorphisms. In the category of topological spaces, the morphisms are continuous maps.

To define a category, one needs four main things. 
\begin{enumerate}
\item Objects.
\item The ``set'' of morphisms from $A$ to $B$ (objects), written $\Mor( A, B)$.
\item Composition. If there is a morphism from $ A$ to $ B$ and from $ B$ to $ C$, then there is a morphism from $ A$ to $ C$.
\item Identity morphisms, sometimes notated $1_{ A}$, in $\Mor( A, A)$.
\end{enumerate}
Categories must satisfy some axioms.
\begin{enumerate}
\item Associativity of composition. If one has 
$$
 A \morphism {f}  B\morphism {g}  C \morphism {h}  D
$$
then $(hg)f = h(gf)$.
\item The identity morphism behaves in the obvious way; namely $f\circ I_{ A} = I_{ A} \circ f = f$.
\end{enumerate}
An interesting example of a category is a group $G$ (not the category of groups). This category has one object, and its morphisms are its elements. The composition of morphisms is the product of $G$. One could take a monoid instead. Morphisms need not be functions from something to something else.

Another example is a poset $P$. \iffalse satisfying $a\leq a$ and 
$$
(a\leqslant b \land b\leqslant c) \implies a\leqslant c.
$$\fi
This category's objects are its elements, and it has one morphism from $a$ to $b$ if $a\leq b$ and none if $a>b$. Composition makes sense since $P$ is partially ordered.

Categories can be generated by drawing points and arrows between them. Consider 
\[
\begin{tikzcd}
A \arrow[r, shift left, ]& B\arrow[l, shift left, ]
\end{tikzcd}
\]
which has $A$ and $B$ as objects, two non-identity morphisms, and two identity morphisms.

There is also a category whose objects are the elements of the natural numbers, morphisms $m\times n$ matrices, and composition is matrix multiplication.

One should note that the set of all sets, groups, or topological spaces does not exist. So how does one define the category? Firstly, one can bound the size of objects by some cardinal $\kappa$. One may also use classes instead of sets. Lastly, one may use \href{https://en.wikipedia.org/wiki/Grothendieck_universe}{\defn{Grothendieck universes}}\index{Grothendieck universe}, which is a process quite similar to bounding the size of the objects by $\kappa$. The easiest method is to ignore the problem altogether.

The fundamental theme of category theory is that one should ignore the internal structure of objects and only look at morphisms.

Recall the notion of an injective map. Suppose (in the category of sets)
$$
g : B\longrightarrow C.
$$
$g$ is injective if $g(b_1) = g(b_2) \implies b_1=b_2$. In the context of category theory, one defines properties in terms of the morphisms, not the objects. So suppose 
\[
\begin{tikzcd}
A \arrow[r, shift right, swap, "f_2"] \arrow[r, shift left, "f_1"]& B \arrow[r, "g"] & C.
\end{tikzcd}
\]
$g$ is a \href{https://en.wikipedia.org/wiki/Monomorphism}{\defn{monomorphism}}\index{monomorphism} if $gf_1 = gf_2 \implies f_1=f_2$. This is the category theoretical analogue of an injective map.

If one has a category, one gets another category by reversing all of the arrows. This is the notion of a dual. There is the category theoretic analogue of a surjective map. So suppose (in the category of sets)
$$
g : B\longleftarrow C
$$
and
\[
\begin{tikzcd}
A & B\arrow[l, shift right, swap, "f_1"] \arrow[l, shift left, "f_2"] & C \arrow[l, "g"].
\end{tikzcd}
\]
$g$ is an \href{https://en.wikipedia.org/wiki/Epimorphism}{\defn{epimorphism}}\index{epimorphism} if $f_1g = f_2g \implies f_1=f_2$. This is the dual of the monomorphism.

Consider the category of rings. Monomorphisms are the same as injective maps of rings, but epimorphisms are not the same as surjective maps of rings. Consider 
$$
\psi: \Z\longrightarrow \Q.
$$
$\psi$ certainly isn't surjective, but it is an epimorphism.

In the category of sets, if an epimorphism is also a monomorphism, then it is an \defn{isomorphism}\index{isomorphism}; namely if $f$ is a morphism from $A$ to $B$, then there is an inverse morphism $g$ from $B$ to $A$ such that $fg$ and $gf$ are the identity map. This is false for general categories (take $\psi$ from the example above with rings, or consider a partially ordered set).


\section{Functors}
A \href{https://en.wikipedia.org/wiki/Functor}{\defn{functor}}\index{functor} is a map from one category to another category. Eilenberg and Mac Lane seemed to have defined functors before categories. They looked at the example of the map from topological spaces to abelian groups defined by $X\longmapsto H_i(X)$. It was noted that when one has a map between topological spaces $X$ and $Y$ there is also a map from $H_i(X)$ to $H_i(Y)$.

Suppose $\cat A$ and $\cat B$ are categories, and $F$ is a functor from $\cat A$ to $\cat B$. $F$ takes each object $a$ in $\cat A$ to an object $F(a)$ in $\cat B$ and each morphism $f$ between objects of $\cat A$ to a morphism $F(f)$ between objects of $\cat B$. $F$ must preserve identities and composition whenever defined.

An example of a functor is the free group, which takes objects in the category of sets to objects in the category of groups. Another example is the \href{https://en.wikipedia.org/wiki/Commutator_subgroup#Abelianization}{\defn{abelianization}}\index{abelianization} of groups, which takes a groups to abelian groups, defined by $G\longmapsto G/ [G,G]$.

The action of a group $G$ on a set $S$ is a functor. The first category has one object and its morphisms are the elements of $G$. The second category is that of sets. The functor takes the object to some set $S$ and every element in $G$ to some map from $S$ to itself. Take the same starting category to the category of vector spaces over $k$ and you get a linear representation of the group, some action of $G$ on a vector space $k$. 

The \href{https://en.wikipedia.org/wiki/Dual_space}{\defn{dual of a vector space}}
$$
V^* = \Hom_k(V,k)
$$
is a functor* from vector spaces to vector spaces. So if $V\longrightarrow W$ is a linear map then we want $V^* \longrightarrow W^*$. Consider 
\[
\begin{tikzcd}
V \arrow[rr, "f"] \arrow[dr] && W\arrow[dl] \\ & k &
\end{tikzcd}
\]
There is no natural way to construct a map from $W$ to $k$, and so the dual of a vector space is, more precisely, a \href{https://en.wikipedia.org/wiki/Functor#Covariance_and_contravariance}{\defn{contravariant functor}}\index{contravariant functor}, since instead we can construct a map $W^* \longrightarrow V^*$. Essentially, this means that the arrows go in the wrong direction. Normal functors are \href{https://en.wikipedia.org/wiki/Functor#Covariance_and_contravariance}{\defn{covariant functors}}\index{covariant functor}. Some people get annoyed by the fact that the \href{https://en.wikipedia.org/wiki/Homology_(mathematics)}{\color{black}homology groups} are covariant functors and \href{https://en.wikipedia.org/wiki/Cohomology}{\color{black}cohomology groups} are contravariant functors, that is, given a map of spaces $X\longrightarrow Y$, we get $H_*(X) \longrightarrow H_*(Y)$ and $H^*(Y) \longrightarrow H^*(X)$.

Another example of a contravariant functor is from the category of topological spaces to the category of abelian groups where a space $X$ maps to the continuous real functions on $X$ $C(X)$, and a map of spaces $X\longrightarrow Y$ corresponds to a map  $C(Y) \longrightarrow C(X)$.

Suppose $\cat C$ and $\cat D$ are categories. A contravariant functor from $\cat C$ to $\cat D$ is the same as a covariant functor from $ {\cat C}\op$ to $\cat D$, where $ {\cat C}\op$ is the \href{https://en.wikipedia.org/wiki/Opposite_category}{\defn{opposite category}}\index{opposite category} of $\cat C$, which is $\cat C$ with every arrow pointing in the opposite direction.

There are also functors of two variables. For example, if $A$ and $B$ are abelian groups, we can consider the homomorphisms from $A$ to $B$, $\Hom(A,B)$, which is also an abelian group. This functor is covariant in $B$, that is, given a map $B_1 \longrightarrow B_2$ we get a map $\Hom(A,B_1) \longrightarrow \Hom(A,B_2)$, and contravariant in $A$, that is, given a map $A_1\longrightarrow A_2$ we get a map $\Hom(A_2,B)\longrightarrow\Hom(A_1,B)$.

Another common example of functors is \href{https://en.wikipedia.org/wiki/Forgetful_functor}{\defn{forgetful functors}}\index{forgetful functor}. For example, a forgetful functor from groups to sets takes a group, forgets its group structure, and maps it to the underlying set.

If you take a model of set theory, then all categories in that model form a category in a bigger model of set theory. In other words, categories form a category. Here, the objects are the categories, the morphisms are the functors, and composition holds. One can then define an isomorphism of categories, but this turns out to be useless. 

\href{https://en.wikipedia.org/wiki/Sheaf_(mathematics)#Presheaves}{\defn{Presheaves}}\index{presheaf} are an example of functors. For example, suppose $X$ is a topological space. Then any open set $U\subseteq X$ can be mapped to $C(U)$, the continuous functions on $U$. If $U\subseteq V$, then there is a restriction map from $C(V) \longrightarrow C(U)$, and if $U\subseteq V\subseteq W$, then we get a composition of restriction maps $C(W) \longrightarrow C(V) \longrightarrow C(U)$. Essentially, for a presheaf, you assign to every open set $U$ some abelian group and for every inclusion of open sets, there is some map between abelian groups that satisfy the above properties. The presheaf functor* maps from the category of open sets of $X$, a partially ordered set, to abelian groups, but this is contravariant.


\section{Natural transformations}
Consider the category of finite dimensional vector spaces over the field $k$. If $V$ is a vector space then $V^*=\Hom_k(V,k)$ is the dual space of $V$. One notices that $V\isomto V^*$ since $V$ is finite dimensional, but this is not natural. Also, $V\isomto V^{**}$, and this is natural; if $v\in V$ and $w = V^*$, we define $v(w) = w(v)$. 

But what does natural mean? For $V$ and $W$ vector spaces, 
\[
\begin{tikzcd}
V \arrow[r] \arrow[d] & W \arrow[d]\\ V^{**} \arrow[r] & W^{**}
\end{tikzcd}
\]
commutes whenever there is a linear transformation from $V$ to $W$. We can write this as
\[
\begin{tikzcd}
F(V) \arrow[r] \arrow[d, "\sim"] & F(W) \arrow[d,"\sim"]\\ G(V) \arrow[r] & G(W)
\end{tikzcd}
\]
where $F(V) = V$ and $G(V) = V^{**}$. Since the maps from $F$ to $G$ are isomorphisms, we call this a natural isomorphism; with arbitrary functors, we call this a \href{https://en.wikipedia.org/wiki/Natural_transformation}{\defn{natural transformation}}\index{natural transformation}.

Consider the determinant, which, for a ring $R$, takes elements in $\GL_n(R) $ to $R^*$. Then
\[
\begin{tikzcd}
\GL_n(R) \arrow[r] \arrow[d] & R^* \arrow[d]\\ \GL_n(S)  \arrow[r] & S^*
\end{tikzcd}
\]
commutes. So the determinant is a natural transformation from $R\longrightarrow \GL_n(R)$ to $R\longrightarrow R^*$.

$V\longrightarrow V^{**}$ is not an isomorphism from the category of vector spaces to itself. Suppose for categories $\cat C$ and $\cat D$
\[
\begin{tikzcd}
\cat C \arrow[r, shift left, "F"] & \cat D. \arrow[l, shift left, "G"]
\end{tikzcd}
\]
Then $F$ is an isomorphism if $FG = I_{\cat D}$ and $GF = I_{\cat C}$. If we have a natural isomorphism from $FG$ to $I_{\cat D}$ and from $GF$ to $I_{\cat C}$, then $F$ and $G$ are \href{https://en.wikipedia.org/wiki/Equivalence_of_categories}{\defn{equivalences of categories}}\index{equivalence of categories}. One now sees that $V\longrightarrow V^{**}$ is an equivalence of categories. The concept of isomorphism is too strong to be useful.

An example of equivalence is \href{https://en.wikipedia.org/wiki/Gelfand_representation}{\defn{Gelfand duality}}\index{Gelfand duality} between \href{https://en.wikipedia.org/wiki/Compact_space}{\color{black}compact Hausdorff spaces} and the opposite category of the \href{https://en.wikipedia.org/wiki/C*-algebra}{\color{black}commutative $C^*$ algebras with identity}. One maps a space $X$ to $C(X)$, which are the continuous complex functions on $X$.  % https://youtu.be/YLZuamSNLvc?t=490

Another example of equivalence is, for $A,B,C$ abelian groups, 
$$
\Hom(A\tensor B, C) \isomto \Hom(A, \Hom(B,C)).
$$

We may also construct the category of functors from $\cat C$ to $\cat D$. Here, the objects are the functors from $\cat C$ to $\cat D$ and the morphisms are natural transformations of functors.

The ``category of categories'' is called the \href{https://en.wikipedia.org/wiki/Strict_2-category}{\color{black}$2$-category}\index{$n$-category}. The $0$-category is a set, the $1$-category is a category with morphisms from $A$ to $B$ a $0$-category, and the $2$-category has its morphisms from $A$ to $B$ a $1$-category.  


\section{Adjoint functors}
\href{https://en.wikipedia.org/wiki/Adjoint_functors}{\defn{Adjoint functors}}\index{adjoint} were introduced by \href{https://en.wikipedia.org/wiki/Daniel_Kan}{\color{black}Kan} in 1958. Suppose we have the forgetful functor $G$ from the category of groups to the category of sets. $F$, the free group functor from the category of sets to groups, and $G$ are adjoint. Consider maps from a set $S$ to a group $H$. Consider
\[
\begin{tikzcd}
H & G(H). \\
F(S) \arrow[u] & S \arrow[u]\arrow[lu]
\end{tikzcd}
\]
We want a natural isomorphism between the functors $\Mor (F(S), H)$ and $\Mor (S, G(H))$. Suppose for categories $\cat A$ and $\cat B$
\[
\begin{tikzcd}
\cat A \arrow[r, shift left, "F"]& \cat B\arrow[l, shift left, "G"].
\end{tikzcd}
\]
If $\Mor_{\cat A}(F(b), a)$ can be identified with $\Mor_{\cat B}(b, G(a))$ by a natural isomorphism, $F$ is left adjoint to $G$ and $G$ is right adjoint to $F$. There is a natural isomorphism between any two left adjoints of a functor.

Suppose $V$ is a vector space over a field $k$ with a bilinear form $(V,V)\longrightarrow k$. Linear operators $F, G$ are adjoint if $(F(b), a)=(b, G(a))$. This is where the word adjoint comes from.

Consider $\Mor_{\cat A}(F(b), a)$ and $\Mor_{\cat B}(b, G(a))$. If $a_1\longrightarrow a_2$ is a morphism, then if
\[
\begin{tikzcd}
\Mor(F(b), a_1) \arrow[r,"\sim"] \arrow[d, ] &\Mor (b, G(a_1)) \arrow[d,]\\ \Mor(F(b), a_2)\arrow[r, "\sim"] & \Mor(b, G(a_2))
\end{tikzcd}
\]
commutes then it is natural in $\cat A$. Similarly, for a morphism $b_1\longrightarrow b_2$, if
\[
\begin{tikzcd}
\Mor(F(b_1), a) \arrow[r, "\sim"] &\Mor (b_1, G(a)) \\ \Mor(F(b_2), a)\arrow[r, "\sim"] \arrow[u]& \Mor(b_2, G(a)) \arrow[u]
\end{tikzcd}
\]
commutes then it is natural in $\cat B$.

Adjoint functors of the form
\[
\begin{tikzcd}
\cat C \arrow[r, shift left, "\mathrm{forget}"]& \cat D\arrow[l, shift left, "\mathrm{free}"]
\end{tikzcd}
\]
are very common. For example, taking $\cat C$ as groups and $\cat D$ as sets, the forgetful functor is the obvious one and the free functor takes the set to its free group; taking $\cat C$ as commutative rings and $\cat D$ as sets, the free functor takes the set to the polynomial ring on the set; taking $\cat C$ as rings and $\cat D$ as \href{https://en.wikipedia.org/wiki/Lie_algebra}{\color{black}Lie algebras}, the free functor takes the Lie algebra to the \href{https://en.wikipedia.org/wiki/Universal_enveloping_algebra}{\color{black}universal enveloping algebra}; taking $\cat C$ to be \href{https://en.wikipedia.org/wiki/Complete_metric_space}{\color{black}complete metric spaces} and $\cat C$ to be metric spaces, the free (left adjoint) functor takes the metric space to its completion; taking $\cat C$ fields and $\cat D$ integral domains, the free functor takes the domain to its field of fractions.

Functors can also have a left and right adjoint. Consider the category of $G$-sets of a group $G$ and the category of sets. Given a set $S$, one gets a functor that takes $S$ to a $G$-set under $S$ with the trivial action. Going from $G$-sets to sets, there is a functor taking a $G$-set $X$ to the set of fixed points $X^G$ and another taking $X$ to the set of orbits of $X$ under $G$. These are left and right adjoint functors of the functor from sets to $G$-sets. Which one is left adjoint, and which one is right adjoint? One can also take a $G$-set $X$ to the corresponding set $X$ by the obvious forgetful functor. There is a functor taking a set $S$ to $G\times S$, a $G$-set, and another taking a set $S$ to the set of all functions from $G$ to $S$. Again, one of these is a left and the other is a right adjoint to the forgetful functor. Which one is which?

If you have an equivalence of categories, then you have two functors which are left and right adjoints to each other.

Right adjoint functors preserve \href{https://en.wikipedia.org/wiki/Limit_(category_theory)}{\color{black}limits} and left adjoint functors preserve \href{https://en.wikipedia.org/wiki/Limit_(category_theory)}{\color{black}colimits}.



\section{Limits and colimits}
Consider topological spaces $X$ and $Y$. We can take the product of the underlying sets $X\times Y$ with a basis of open sets $U\times V$.

In category theory, the product is a universal construction. 

There exist natural projection maps from $X\times Y$ to $X$ and $Y$. Suppose $Z$ is a space with maps to $X$ and $Y$. Then $Z$ factorizes through $X\times Y$ in a unique way. Suppose $P_1,P_2$ are products of $X $ and $Y$. They each have a universal map to $X$ and $Y$, and since $P_2$ is universal, the map from $P_1$ to $X$ and $Y$ factors through $P_2$, and since $P_1$ is universal, one gets a map the other way. Products are unique up to isomorphism in any category provided they exist.

We can take products of an arbitrary number of objects $X_1,\hdots,X_n$ where the product $X_1\times\cdots\times X_n$ maps to each $X_i$, and it is universal; that is, if $S$ maps to each $X_i$ then it factors through the product uniquely. If the $X_i$s are topological spaces (avoiding the box topology) then we take a basis $U_1\times\cdots\times U_n$ open in each $X_i$ where all but a finite number have $U_i=X_i$.

Suppose we have
\[
\begin{tikzcd}
X\arrow[r, "f"] &A \arrow[r, shift left, "g"]\arrow[r, shift right, swap, "h"]& B
\end{tikzcd}
\]
for objects $X, A,B$ and morphisms $f,g,h$. $X$ is an \href{https://en.wikipedia.org/wiki/Equaliser_(mathematics)}{\defn{equalizer}}\index{equalizer} if $gf=hf$. If $S$ is a space mapping to $A$ such that $g,h$ are ``equal'' then there is a unique map from $S$ to $X$ making 
\[
\begin{tikzcd}
X\arrow[r, "f"] &A \arrow[r, shift left, "g"]\arrow[r, shift right, swap, "h"]& B\\
S \arrow[ru] \arrow[u, dashed]
\end{tikzcd}
\]
commute. 

For $A, B$ abelian groups, if we have
\[
\begin{tikzcd}
\ker f \arrow[r] &A \arrow[r, shift left, "f"]\arrow[r, shift right, swap, "0"]& B
\end{tikzcd}
\]
then $\ker f$ is an equalizer.

We can also consider \href{https://en.wikipedia.org/wiki/Pullback_(category_theory)}{\defn{pullbacks}}\index{pullback} or \href{https://en.wikipedia.org/wiki/Pullback_(category_theory)}{\defn{fibered products}}\index{fibered product}. If objects $X,Y$ both map to $Z$, then we have the product $X\times_Z Y$ that uniquely maps to $X$ and $Y$, which means that for any object $S$ mapping to $X$ and $Y$ there is a unique map from $S $ to $X\times_Z Y$:
\[
\begin{tikzcd}
S \arrow [dr] \arrow[drr, shift left] \arrow[ddr] &&\\
&X\times_Z Y \arrow[r] \arrow[d]& Y\arrow[d]\\
& X\arrow[r]&Z
\end{tikzcd}
\]
This shows up often in algebraic geometry as a \href{https://en.wikipedia.org/wiki/Base_change_theorems}{\color{black}base change map}\index{base change map}.

Suppose $Y=TM$ is a \href{https://en.wikipedia.org/wiki/Tangent_space}{\color{black}tangent space} of a manifold projecting to the manifold $M$ and $p$ is a point on $M$. Then the pullback is the tangent space at $p$, $T_p$:
\[
\begin{tikzcd}
T_p \arrow[r] \arrow[d]& Y \arrow[d]\\
p\arrow[r]&M.
\end{tikzcd}
\]
The pullback for sets is the subset of $X\times Y$ of points $(x,y)$ with the same image in $Z$.

An \defn{inverse system}\index{inverse system} is a sequence of objects $(A_n)$ and a sequence of morphisms $(\psi_n)$ 
\[
\begin{tikzcd}
\cdots \arrow[r]  & A_{n+1}\arrow[r, "\psi_n"] & A_n \arrow[r] &\cdots \arrow[r]  & A_2 \arrow[r, "\psi_1"] & A_1.
\end{tikzcd}
\]
The \href{https://en.wikipedia.org/wiki/Inverse_limit}{\defn{inverse limit}}\index{inverse limit} 
\[
A = \varprojlim (A_n, \psi_n)
\]
is the subset of the direct product $\prod_n A_n$ consisting of the sequences $a=(a_n)$ such that $\psi_n(a_{n+1}) = a_n$ for all $n\ge 1$. For every $n\ge 1$, the \defn{projection} $\pi_n : A \longrightarrow A_n$ is given by $a \longmapsto a_n$.

The $p$-adic numbers $\Z_p$ make every triangle in 
\[
\begin{tikzcd}
&&\Z_p \arrow[dll]\arrow[dl] \arrow[d]\arrow[dr] \arrow[drr] &&\\
\Z/p & \Z/p^2 \arrow[l] & \Z/p^3 \arrow[l] & \Z/p^4\arrow[l]  & \cdots \arrow[l]
\end{tikzcd}
\]
commute, and this property is universal: if $S$ maps to all of the $\Z/p^i$, then it factors through $\Z_p$. This is an example of an inverse limit.

Suppose $\cat C$ and $\cat J$ are categories. A \href{https://en.wikipedia.org/wiki/Diagram_(category_theory)}{\defn{diagram}}\index{diagram} of shape $\cat J$ in $\cat C$ is a functor 
$$
F : \cat J \longrightarrow \cat C.
$$
$\cat J$ can be thought of as an index category and $F$ as indexing a collection of objects and morphisms in $\cat C$ patterned on $\cat J$. A \href{https://en.wikipedia.org/wiki/Cone_(category_theory)}{\color{black}cone}\index{cone} to $F$ is an object $N$ of $\cat C$ with a family
$$
\psi_X : N \longrightarrow F(X)
$$
of morphisms indexed by the objects $X$ of $\cat J$ such that for every morphism $f:X\longrightarrow Y$ in $\cat J$ we have $F(f) \circ \psi_X = \psi_Y$.

A \href{https://en.wikipedia.org/wiki/Limit_(category_theory)}{\defn{limit}}\index{limit} of the diagram $F:\cat J\longrightarrow\cat C$ is a cone $(L,\phi)$ to $F$ such that for every other cone $(N,\psi)$ to $F$ there exists a unique morphism $\pi:N\longrightarrow L$ such that $\phi_X \circ \pi = \psi_X$ for all $X$ in $\cat J$:
 \[
\begin{tikzcd}[row sep = large]
& N \arrow[d, "\pi", dashed] \arrow[ddr, bend left,"\psi_Y"]   \arrow[ddl, bend right,"\psi_X",swap] & \\
& L\arrow[dr, "\phi_X"]  \arrow[dl,swap, "\phi_Y"] & \\
 F(X) \arrow[rr,swap, "F(f)"] & &  F(Y) 
\end{tikzcd}
\]
Limits are unique up to isomorphism if they exist. If $\cat J$ is two points, then the limit would be product of the two points; if $\cat J$ has two objects with two morphisms between them, the limit is the equalizer; if we take $\cat J$ to be the empty category, then the limit is the \href{https://en.wikipedia.org/wiki/Initial_and_terminal_objects}{\defn{final object}}\index{final object}\index{terminal object}.

We also have \href{https://en.wikipedia.org/wiki/Limit_(category_theory)}{\defn{colimits}}\index{colimit}, which can be thought of as the dual of limits. If $\cat J$ is two points, then the colimit is the \href{https://en.wikipedia.org/wiki/Coproduct}{\defn{coproduct}}\index{coproduct}. For example, in groups, this is the \href{https://en.wikipedia.org/wiki/Free_product}{\defn{free product}}\index{free product} of two groups $G*H$: the biggest group generated by subgroups $G$ and $H$. For commutative rings, the coproduct is the tensor product of rings $R_1\tensor R_2$. Notice that for abelian groups the coproduct is the same as the product. We can also take infinite coproducts, which, for abelian groups, are not equal to infinite products. Infinite coproducts of abelian groups are direct sums. If $\cat J$ is a \href{https://en.wikipedia.org/wiki/Pushout_(category_theory)}{\defn{pushout}}\index{pushout} or fibered coproduct\index{fibered coproduct} (the dual of a pullback), for groups, we get the \href{https://en.wikipedia.org/wiki/Free_product#Generalization:_Free_product_with_amalgamation}{\color{black}amalgamated product}\index{amalgamated product}. If $G,H$ are groups with a common subgroup $X$ then $G*_X H$ is the biggest group generated by $G$ and $H$ while identifying the subgroup $X$ of $G$ with the subgroup $X$ of $H$. We have the concept of the \href{https://en.wikipedia.org/wiki/Coequalizer}{\defn{coequalizer}}\index{coequalizer} For abelian groups, the example of the coequalizer would be the quotient
\[
\begin{tikzcd}
A \arrow[r, shift left, "f"]\arrow[r, shift right, swap, "0"]& B\arrow[r] & B/(\im A)
\end{tikzcd}
\] 
the \href{https://en.wikipedia.org/wiki/Cokernel}{\defn{cokernel}}\index{cokernel}.

Suppose in sets we have $A_1 \subseteq A_2 \subseteq A_3 \subseteq A_4$. Then an example of the \href{https://en.wikipedia.org/wiki/Direct_limit}{\defn{direct limit}}\index{direct limit} would be the union of the $A_i$s. 

If $\cat J$ is the empty category then the colimit is the \href{https://en.wikipedia.org/wiki/Initial_and_terminal_objects}{\defn{initial object}}\index{initial object}.

Right adjoints, often forgetful functors, preserve limits. Left adjoints, often ``free'' functors, preserve colimits. For example, we have
\[
\begin{tikzcd}
\catn{Set} \arrow[r, shift left, "\mathrm{free}"]& \catn{Grp}. \arrow[l, shift left, "\mathrm{forget}"]
\end{tikzcd}
\]
So taking the forgetful functor should preserve limits. One gets
$$
\mathrm{Set}(G_1\times G_2) = \mathrm{Set}(G_1) \times \mathrm{Set}(G_2)
$$
and
$$
\mathrm{Set}(G_1* G_2) \neq \mathrm{Set}(G_1) \cup \mathrm{Set}(G_2).
$$
Also,
$$
\mathrm{Free}(S_1 \cup S_2) = \mathrm{Free}(S_1)*\mathrm{Free}(S_2)
$$
and
$$
\mathrm{Free}(S_1\times S_2) \neq \mathrm{Free}(S_1) \times \mathrm{Free}(S_2).
$$
\href{https://en.wikipedia.org/wiki/Formal_criteria_for_adjoint_functors}{\defn{Freyd's adjoint functor theorem}}\index{Freyd's adjoint functor theorem} states that if $F$ preserves limits then $F$ often has a left adjoint. If one considers the forgetful functor from \href{https://en.wikipedia.org/wiki/Complete_Boolean_algebra}{\defn{complete Boolean algebras}}\index{complete Boolean algebra} to sets, it has no left adjoint.

Recall from homological algebra that $F$ is a \href{https://en.wikipedia.org/wiki/Exact_functor}{\defn{right exact functor}}\index{right exact functor} if for an exact sequence of groups
$$
\begin{tikzcd}
A \arrow[r] & B\arrow[r]& C\arrow[r] & 0
\end{tikzcd}
$$
then
$$
\begin{tikzcd}
F(A) \arrow[r] & F(B)\arrow[r]& F(C)\arrow[r] & 0
\end{tikzcd}
$$
is an exact sequence of groups. One notices that $C$ is the colimit (coequalizer) of the diagram above. In particular, a left adjoint functor (preserving colimits) is right exact and a right adjoint functor is left exact. Suppose $F(A) = A \tensor Y$. We want to show that this is right exact. One notices that $\Mor(X\tensor Y, Z) \isomto \Mor(X, \Hom(Y, Z))$, so $A\tensor Y$ is left adjoint to $\Hom (Y,A)$, and so it follows that tensor products preserve right exactness. 

``One of the purposes of category theory is to make trivial results trivial.''

Suppose we have
$$
\begin{tikzcd}
\Z \arrow[r, "\times 2"] & \Z\arrow[r, "\times 2"]& \Z\arrow[r, "\times 2"] & \cdots
\end{tikzcd}
$$
The colimit of this is the set of rationals of the form $n/2^m$. Perhaps we could simplify this diagram to 
$$
\begin{tikzcd}
\Z \arrow[loop, "\times 2"] 
\end{tikzcd}
$$
The colimit of this diagram is $0$. What has gone wrong is that the first diagram is not a subcategory. You do not take limits of colimits over subcategories. 

\section{Monoidal categories}
The category of sets has a product $\times$. We can define monoids in the normal way. They have a map $A\times A \longrightarrow A$ and $1\longrightarrow A$. The category of abelian groups has a tensor product $\tensor$. We can define a rings in the normal way, and we have a map $A\tensor A \longrightarrow A$ and $\Z \longrightarrow A$. 

\href{https://en.wikipedia.org/wiki/Monoidal_category}{\defn{Monoidal categories}}\index{monoidal category} are a generalization of these two examples. It should have some \href{https://en.wikipedia.org/wiki/Functor#Bifunctors_and_multifunctors}{\defn{bifunctor}}\index{bifunctor} $\tensor$ and some object playing the role of the identity $1$. We want to define monoids in this category to be objects with a map $A\tensor A\longrightarrow A$ and $1\longrightarrow A$. What conditions are there on the bifunctor?

Suppose we put the condition of associativity; namely $(A\tensor B)\tensor C = A \tensor (B\tensor C)$ and $1\tensor A = A = A\tensor 1$. This doesn't really work, for instance, with the tensor product (it is isomorphic), but we will assume it anyway. We call a category satisfying these conditions a \href{https://en.wikipedia.org/wiki/Monoidal_category}{\defn{strict monoidal category}}\index{strict monoidal category}.

One example of this is any monoid; this category has $1$ element and the morphisms are the elements of the monoid. Another examples is the category of finite ordinals with order-preserving maps, which has elements $0=\emptyset$, $1 = \{0\}$, $2 = \{0,1\}$, $3 = \{0,1,2\}$, etc. with $2\tensor 3 = 5$ (for example) and $0$ the identity. The object $1$ is the monoid, so there is a map $1\tensor 1 = 2\longrightarrow 1$. Also
$$
n \isomto \underbrace{1 \tensor \cdots \tensor 1 }_{n \textrm{ times}}.
$$
This is a universal example. Another example: fix a category $\cat C$ and we get a new category with objects functors $\cat C\longrightarrow\cat C$ and morphisms as natural transformations. We define $\tensor$ as composition of functors, and we get a strict monoidal category, where the monoids are \href{https://en.wikipedia.org/wiki/Monad_(category_theory)}{\defn{monads}}\index{monad}.

In a monoidal category, we need to be given an isomorphism $(A\tensor B)\tensor C \isom A\tensor(B\tensor C)$ and $A \tensor 1 \isom A $ and $1\tensor A\isom A$. For abelian groups, we could take $(a\tensor b)\tensor c \longrightarrow a\tensor (b\tensor c)$, but why can't we take $(a\tensor b)\tensor c \longrightarrow -a\tensor (b\tensor c)$? The problem here is a coherence condition: consider 
 \[
\begin{tikzcd}
A \tensor (B \tensor (C \tensor D)) \arrow[r]  & (A \tensor B)\tensor (C \tensor D) \arrow[r] & ((A\tensor B)\tensor C)\tensor D\arrow [d] \\
A\tensor ((B\tensor C)\tensor D) \arrow[u]&& (A\tensor (B\tensor C))\tensor D \arrow[ll]
\end{tikzcd}
\]
This diagram does not commute for $(a\tensor b)\tensor c \longrightarrow -a\tensor (b\tensor c)$. We also have to consider the coherence condition that
 \[
\begin{tikzcd}
A\tensor (1 \tensor B) \arrow[rr]  \arrow [rd]&& (A\tensor 1) \tensor B \arrow[ld]\\
& A\tensor B
\end{tikzcd}
\]
commutes. One imagines that there are more convoluted coherence conditions to be proved for $5$ or more tensor products, but it turns out that if one proves these two conditions, no more needs to be proved. % https://youtu.be/cU8E5O8oV_g?t=710v

Examples of monoidal categories include sets, which has product defined in the normal way and monoids also defined in the normal way. For abelian groups, the product is the tensor product and the monoids are rings. For categories, the product is the product of categories and the monoids are strict monoidal categories.

We also have \href{https://en.wikipedia.org/wiki/Symmetric_monoidal_category}{\defn{symmetric monoidal categories}}\index{symmetric monoidal category}. The usual tensor product has $\tau:A\tensor B \isomto B \tensor A$. If we want to define a commutative monoid, we could have that
 \[
\begin{tikzcd}
A\tensor A \arrow[rr, "\tau"]  \arrow [rd]&& A\tensor A \arrow[ld]\\
& A
\end{tikzcd}
\]
commutes. Of course, we don't always have this extra (symmetric) structure (take composition). We have more coherence conditions; namely, a hexagon diagram for associativity. We notice that if we have $a\tensor b\longrightarrow -b\tensor a$, the hexagon does not commute, so this is not a good way to define the isomorphism. We also need
 \[
\begin{tikzcd}
A \tensor B \arrow[r] \arrow[rr, bend right, "\mathrm{id}",swap] & B \tensor A\arrow [r] &A\tensor B
\end{tikzcd}
\]
to hold. We also find that
$$
\underbrace{A \tensor \cdots \tensor A }_{n \textrm{ times}}
$$
is acted on by the symmetric group on $n$ letters $S_n$. If we drop the identity condition, we get \href{https://en.wikipedia.org/wiki/Braided_monoidal_category}{\defn{braided monoidal categories}}\index{braided monoidal category}; they are closely related to \href{https://en.wikipedia.org/wiki/Braid_group}{\defn{braid groups}}\index{braid group}.

Let's consider a non-trivial example of a symmetric monoidal category. Take the category of abelian groups with the usual tensor product $\tensor$. We need an isomorphism 
$$
(A\tensor B)\tensor C \longrightarrow A \tensor (B\tensor C) : (a\tensor b)\tensor c \longmapsto a \tensor (b\tensor c) 
$$
and an isomorphism
$$
A \tensor B \longrightarrow B\tensor A : a\tensor b\longmapsto b\tensor a.
$$ 
We found that $a\tensor b\longmapsto -b\tensor a$ doesn't work, but it almost works. Consider $\Z/2\Z$--graded abelian groups and then
$$
a\tensor b \longmapsto (-1)^{\deg a  \deg b}b\tensor a.
$$
We now have two distinct isomorphisms to choose from from $A\tensor B$ to $B\tensor A$. The first gives you the ``usual'' commutative rings, and this considered isomorphism gives us \href{https://en.wikipedia.org/wiki/Supercommutative_algebra}{\defn{supercommutative rings}}\index{supercommutative ring} where 
$$
ab = (-1)^ {\deg a  \deg b}b a.
$$
We get two distinct monoidal categories with two different isomorphisms between $A\tensor B$ and $B\tensor A$. We can translate this notion to Lie algebras. The usual commutation relation has $[a,b] = ab-ba$. In quantum mechanics, the position and momentum operators satisfy $AB-BA=1$. When physicists started working with fermions, they got annihilation operators where $AB+BA =1$, which is actually $AB - (-1)^{\deg A \deg B} BA = 1$, the supercommutator. They were working in the sort of twisted, ``super'' version of abelian groups described above.

\section{Yoneda's lemma}
Suppose $A \in \cat C$. We can define a functor $h_A:\cat C \longrightarrow \catn{Set}$ given by $B \longmapsto \Mor(B,A)$.
\newpage
\bibliographystyle{amsalpha}
\bibliography{bib}
\printindex
\end{document}